\usepackage[utf8]{inputenc} % input encoding
\usepackage[OT1, OT2, T1]{fontenc} % font encoding
\usepackage{amsmath, amsfonts, amssymb, amsthm, latexsym, mathrsfs, mathtools, stmaryrd} % basic math fonts, symbols, and commands
\usepackage{array} % expanded functionality of matrix and tabular environments
\usepackage[UKenglish]{babel} % automatically fixes some character things
\usepackage{bm} % bold math font
\usepackage{booktabs} % nicer tables
\usepackage{cancel} % cancelling out variables
\usepackage[autostyle, english=british]{csquotes} % use literal quotes as quotes
\usepackage{emptypage} % makes a truly empty page (this is, no headers or footers)
\usepackage[shortlabels]{enumitem} % more options for enumerate
\usepackage{etoolbox} % tools for creating commands
\usepackage{fancyhdr} % fancy headers
\usepackage{float} % positioning of floating objects
\usepackage{listings} % code blocks
\usepackage{lstautogobble} % code blocks options
\usepackage{marginnote} % for when marginpar doesn't work
\usepackage{multicol} % allows for multiple columns within a single table column
\usepackage{multirow} % allows for multiple rows within a single table row
\usepackage{parskip} % no indents
\usepackage{pgfplotstable} % figure, color, graphicx support including tikz
\usepackage{polynom} % tools for manipulating polynomials
\usepackage{siunitx} % SI units
\usepackage{subcaption} % allows captions
\usepackage{systeme} % creating systems of equations
\usepackage{gensymb, textcomp} % generic text symbols
\usepackage{thmtools} % tools for making theorems
\usepackage{tikz-cd} % draw commutative diagrams
\usepackage{titlesec} % tools for changing titles of sections
\usepackage{tocloft} % allows me to create a new table of contents
\usepackage[normalem]{ulem} % allows for strike-through text
\usepackage[dvipsnames]{xcolor} % colours and coloured text
\usepackage{xifthen} % extended if then else commands
\usepackage{xstring} % tools for manipulating strings 
\usepackage[all]{xy} % tools for drawing basic diagrams
\usepackage[framemethod=TikZ]{mdframed}
\usepackage{bookmark} % hyperref but with fixes 
\usepackage{url} % allows for external urls
\usepackage[hmargin=1.4in,vmargin=1.4in,marginparwidth=0.75in]{geometry} % margins
\usepackage{href-ul} % underline hyperlinks

% =================================================================================================

% custom colours
\definecolor{grey}{rgb}{0.5,0.5,0.5}
\definecolor{mauve}{rgb}{0.58,0,0.82}
\definecolor{checkmarkgreen}{HTML}{009900}
\definecolor{hanpurple}{rgb}{0.32, 0.09, 0.98}
\definecolor{atomictangerine}{rgb}{1.0, 0.6, 0.4}
\definecolor{lightgrey}{gray}{0.92}
\definecolor{stackgrey}{gray}{0.95}

% =================================================================================================

% forces PDF version
\pdfminorversion=7
\pdfsuppresswarningpagegroup=1

% =================================================================================================

% margins and text dimensions
\setlength{\textheight}{9in}
\setlength{\textwidth}{6.5in}
\setlength{\topmargin}{0.0in}
\setlength{\leftmargin}{0.0in}
\setlength{\oddsidemargin}{0.0in}
\reversemarginpar

% =================================================================================================

% titles, sections, and toc stuff
\newlistof{lectures}{lec}{Lectures} % define the lecture section of the ToC 

\setcounter{tocdepth}{3}
\setcounter{secnumdepth}{1}

% =================================================================================================

% makes things appear underneath certain operators by default by appending \limits to them
\apptocmd{\lim}{\limits}{}{}
\apptocmd{\sum}{\limits}{}{}
\apptocmd{\prod}{\limits}{}{}
\apptocmd{\bigwedge}{\limits}{}{}
\apptocmd{\bigvee}{\limits}{}{}
\apptocmd{\bigcap}{\limits}{}{}
\apptocmd{\bigcup}{\limits}{}{}
\apptocmd{\bigoplus}{\limits}{}{}
\apptocmd{\bigotimes}{\limits}{}{}
\apptocmd{\max}{\limits}{}{}
\apptocmd{\min}{\limits}{}{}
\apptocmd{\sup}{\limits}{}{}
\apptocmd{\inf}{\limits}{}{}

% =================================================================================================

% changing the looks of some variables to better looking versions
\let\epsilon\varepsilon
\let\iff\Leftrightarrow
\let\implies\Rightarrow
\let\impliedby\Leftarrow

% =================================================================================================

% set some figure settings
\pgfplotsset{compat=newest}
\usepgfplotslibrary{fillbetween}
\usetikzlibrary{patterns}
\usetikzlibrary{intersections}

% =================================================================================================

% make the @ symbol a usable letter for the next bit
\makeatletter

% indent subsections in the ToC
\renewcommand*\l@subsection{\@dottedtocline{2}{5.5em}{4.5em}}

% =================================================================================================

% package settings

\lstset{
    autogobble,
    basewidth=0.5em,
    basicstyle={\small\ttfamily},
    belowskip=1mm,
    breakatwhitespace=true,
    breaklines=true,
    captionpos=b,
    escapeinside={(*}{*)},
    frame=t,
    framexleftmargin=2mm,
    keepspaces=true,
    numbers=left,
    numbersep=3mm,
    numberstyle=\scriptsize\color{grey},
    showspaces=false,
    showstringspaces=false,
    showtabs=false,
    tabsize=2,
    upquote=true,
    xleftmargin=4mm,
}

\@ifclasswith{report}{nocolor}
{
    \hypersetup{
        colorlinks,
        linkcolor={black!50!black},
        citecolor={black!50!black},
        urlcolor={black!80!black}
    }
}
{
    \hypersetup{
        colorlinks,
        linkcolor={red!50!black},
        citecolor={blue!50!black},
        urlcolor={blue!80!black}
    }
    \lstset{
        backgroundcolor=\color{stackgrey},
        commentstyle=\color{checkmarkgreen},
        keywordstyle=\color{blue},
        stringstyle=\color{mauve},
    }
}

% =================================================================================================

% custom commands
\@ifclasswith{report}{nocolor}
{
    % incorrect text
    \newcommand{\red}[1]{{\sout{#1}}}
    % correct text
    \newcommand{\green}[1]{#1}
}
{
    % incorrect text
    \newcommand{\red[1]}{{\color{red}{#1}}}
    % correct text
    \newcommand{\green[1]}{{\color{checkmarkgreen}{#1}}}
}

% big cdot
\newcommand*{\bigcdot}{}% Check if undefined
\DeclareRobustCommand*{\bigcdot}{%
    \mathbin{\mathpalette\bigcdot@{}}%
}
\newcommand*{\bigcdot@scalefactor}{.5}
\newcommand*{\bigcdot@widthfactor}{1.15}
\newcommand*{\bigcdot@}[2]{%
    % #1: math style
    % #2: unused
    \sbox0{$#1\vcenter{}$}% math axis
    \sbox2{$#1\cdot\m@th$}%
    \hbox to \bigcdot@widthfactor\wd2{%
        \hfil
        \raise\ht0\hbox{%
            \scalebox{\bigcdot@scalefactor}{%
                \lower\ht0\hbox{$#1\bullet\m@th$}%
            }%
        }%
        \hfil
    }%
}

% red -> green, for marking purposes
\newcommand{\correction}[2]{\ensuremath{\:}{\red{#1}}\ \ensuremath{\to }\ {\green{#2}}\ensuremath{\:}}

% horizontal rule
\newcommand{\hr}{\par \noindent\rule[0.5ex]{\linewidth}{0.5pt} \par}

% =================================================================================================

% basic margin setup
\mdfsetup{skipabove=12pt, skipbelow=0em, innertopmargin=5pt, innerbottommargin=6pt}

% basic style setup
\theoremstyle{definition}

% custom environments
\@ifclasswith{report}{nocolor}
{
    \declaretheoremstyle[
        headfont=\bfseries\sffamily, bodyfont=\normalfont,
        mdframed={
                nobreak
            }
    ]{purplebox}
    \declaretheoremstyle[
        headfont=\bfseries\sffamily, bodyfont=\normalfont,
        mdframed={
                rightline=false, topline=false, bottomline=false,
            }
    ]{purpleline}
    \declaretheoremstyle[
        headfont=\bfseries\sffamily, bodyfont=\normalfont,
        mdframed={
                rightline=false, topline=false, bottomline=false,
            }
    ]{orangebox}
    \declaretheoremstyle[
        headfont=\bfseries\sffamily, bodyfont=\normalfont,
        mdframed={
                rightline=false, topline=false, bottomline=false,
            }
    ]{orangeline}
    \declaretheoremstyle[
        headfont=\bfseries\sffamily, bodyfont=\normalfont,
        mdframed={
                nobreak
            }
    ]{greenbox}
    \declaretheoremstyle[
        headfont=\bfseries\sffamily, bodyfont=\normalfont,
        mdframed={
                rightline=false, topline=false, bottomline=false,
            }
    ]{lightgreenbox}
    \declaretheoremstyle[
        headfont=\bfseries\sffamily, bodyfont=\normalfont,
        mdframed={
                nobreak
            }
    ]{greenline}
    \declaretheoremstyle[
        headfont=\bfseries\sffamily, bodyfont=\normalfont,
        mdframed={
                nobreak
            }
    ]{bluebox}
    \declaretheoremstyle[
        headfont=\bfseries\sffamily, bodyfont=\normalfont,
        mdframed={
                rightline=false, topline=false, bottomline=false,
            }
    ]{blueline}
    \declaretheoremstyle[
        headfont=\bfseries\sffamily, bodyfont=\normalfont,
        mdframed={
                nobreak
            }
    ]{redbox}
    \declaretheoremstyle[
        headfont=\bfseries\sffamily, bodyfont=\normalfont,
        numbered=no,
        mdframed={
                rightline=false, topline=false, bottomline=false,
            },
        qed=\qedsymbol
    ]{proofbox}
    \declaretheoremstyle[
        headfont=\bfseries\sffamily, bodyfont=\normalfont,
        numbered=no,
        headpunct=,
        postheadspace=0px,
        mdframed={
                rightline=false, topline=false, bottomline=false,
            },
        qed=\qedsymbol
    ]{proofboxn}
    \declaretheoremstyle[
        headfont=\bfseries\sffamily, bodyfont=\normalfont,
        numbered=no,
        mdframed={
                rightline=false, topline=false, bottomline=false,
            },
    ]{explanationbox}
}
{
    \declaretheoremstyle[
        headfont=\bfseries\sffamily\color{hanpurple!70!black}, bodyfont=\normalfont,
        mdframed={
                linewidth=2pt,
                rightline=false, topline=false, bottomline=false,
                linecolor=hanpurple, backgroundcolor=hanpurple!10,
            }
    ]{purplebox}
    \declaretheoremstyle[
        headfont=\bfseries\sffamily\color{hanpurple!70!black}, bodyfont=\normalfont,
        mdframed={
                linewidth=2pt,
                rightline=false, topline=false, bottomline=false,
                linecolor=hanpurple
            }
    ]{purpleline}
    \declaretheoremstyle[
        headfont=\bfseries\sffamily\color{atomictangerine!70!black}, bodyfont=\normalfont,
        mdframed={
                linewidth=2pt,
                rightline=false, topline=false, bottomline=false,
                linecolor=atomictangerine, backgroundcolor=atomictangerine!10,
            }
    ]{orangebox}
    \declaretheoremstyle[
        headfont=\bfseries\sffamily\color{atomictangerine!70!black}, bodyfont=\normalfont,
        mdframed={
                linewidth=2pt,
                rightline=false, topline=false, bottomline=false,
                linecolor=atomictangerine
            }
    ]{orangeline}
    \declaretheoremstyle[
        headfont=\bfseries\sffamily\color{ForestGreen!70!black}, bodyfont=\normalfont,
        mdframed={
                linewidth=2pt,
                rightline=false, topline=false, bottomline=false,
                linecolor=ForestGreen, backgroundcolor=ForestGreen!10,
            }
    ]{greenbox}
    \declaretheoremstyle[
        headfont=\bfseries\sffamily\color{ForestGreen!70!black}, bodyfont=\normalfont,
        mdframed={
                linewidth=2pt,
                rightline=false, topline=false, bottomline=false,
                linecolor=ForestGreen, backgroundcolor=ForestGreen!5,
            }
    ]{lightgreenbox}
    \declaretheoremstyle[
        headfont=\bfseries\sffamily\color{ForestGreen!70!black}, bodyfont=\normalfont,
        mdframed={
                linewidth=2pt,
                rightline=false, topline=false, bottomline=false,
                linecolor=ForestGreen
            }
    ]{greenline}
    \declaretheoremstyle[
        headfont=\bfseries\sffamily\color{NavyBlue!70!black}, bodyfont=\normalfont,
        mdframed={
                linewidth=2pt,
                rightline=false, topline=false, bottomline=false,
                linecolor=NavyBlue, backgroundcolor=NavyBlue!10,
            }
    ]{bluebox}
    \declaretheoremstyle[
        headfont=\bfseries\sffamily\color{NavyBlue!70!black}, bodyfont=\normalfont,
        mdframed={
                linewidth=2pt,
                rightline=false, topline=false, bottomline=false,
                linecolor=NavyBlue
            }
    ]{blueline}
    \declaretheoremstyle[
        headfont=\bfseries\sffamily\color{RawSienna!70!black}, bodyfont=\normalfont,
        mdframed={
                linewidth=2pt,
                rightline=false, topline=false, bottomline=false,
                linecolor=RawSienna, backgroundcolor=RawSienna!10,
            }
    ]{redbox}
    \declaretheoremstyle[
        headfont=\bfseries\sffamily\color{RawSienna!70!black}, bodyfont=\normalfont,
        numbered=no,
        mdframed={
                linewidth=2pt,
                rightline=false, topline=false, bottomline=false,
                linecolor=RawSienna, backgroundcolor=RawSienna!5,
            },
        qed=\qedsymbol
    ]{proofbox}
    \declaretheoremstyle[
        headfont=\bfseries\sffamily\color{RawSienna!70!black}, bodyfont=\normalfont,
        numbered=no,
        headpunct=,
        postheadspace=0px,
        mdframed={
                linewidth=2pt,
                rightline=false, topline=false, bottomline=false,
                linecolor=RawSienna, backgroundcolor=RawSienna!5,
            },
        qed=\qedsymbol
    ]{proofboxn}
    \declaretheoremstyle[
        headfont=\bfseries\sffamily\color{NavyBlue!70!black}, bodyfont=\normalfont,
        numbered=no,
        mdframed={
                linewidth=2pt,
                rightline=false, topline=false, bottomline=false,
                linecolor=NavyBlue, backgroundcolor=NavyBlue!5,
            },
    ]{explanationbox}
}

\declaretheorem[numbered=no, style=blueline, name=Note]{note} % note
\declaretheorem[numbered=no, style=blueline, name=Remark]{remark} % remark
\declaretheorem[numbered=no, style=blueline, name=Idea]{idea} % idea
\declaretheorem[numbered=no, style=greenbox, name=Problem]{problem} % problem
\declaretheorem[numbered=no, style=greenbox, name=Question]{question} % problem
\declaretheorem[numbered=no, style=greenbox, name=Recall]{recall} % recall
\declaretheorem[numbered=no, style=orangebox, name=Notice]{notice} % notice
\declaretheorem[numbered=no, style=orangebox, name=Observation]{observation} % observation
\declaretheorem[numbered=no, style=orangebox, name=Investigation]{investigation} % investigation
\declaretheorem[numbered=no, style=orangebox, name=Goal]{goal} % goal
\declaretheorem[numbered=no, style=orangebox, name=Summary]{summary} % summary
\declaretheorem[numbered=no, style=orangeline, name=Conjecture]{conjecture} % conjecture
\declaretheorem[numbered=no, style=orangeline, name=Intuition]{iintuition} % intuition
\newenvironment{intuition}{\vspace{-10pt}\begin{iintuition}}{\end{iintuition}}
\declaretheorem[numbered=no, style=purpleline, name=Notation]{notation} % notation
\declaretheorem[numbered=no, style=purpleline, name=Convention]{convention} % convention

\declaretheorem[numberwithin=chapter, style=bluebox, name=Example]{example} % example
\declaretheorem[numberwithin=chapter, style=bluebox, name=Exercise]{exercise} % exercise
\declaretheorem[numberwithin=chapter, style=orangebox, name=Property]{property} % property
\declaretheorem[numberwithin=chapter, style=purplebox, name=Definition]{definition} % definition
\declaretheorem[numberwithin=chapter, style=purplebox, name=Algorithm]{algorithm} % algorithm
\declaretheorem[numberwithin=chapter, style=redbox, name=Claim]{claim} % claim
\declaretheorem[numberwithin=chapter, style=redbox, name=Corollary]{corollary} % corollary
\declaretheorem[numberwithin=chapter, style=redbox, name=Lemma]{lemma} % lemma
\declaretheorem[numberwithin=chapter, style=redbox, name=Proposition]{proposition} % proposition
\declaretheorem[numberwithin=chapter, style=redbox, name=Theorem]{theorem} % theorem

\newcounter{proof} % creates a variable that counts the number of proof environments
\counterwithin{proof}{chapter}
\declaretheorem[style=proofbox, name=Proof]{replacementproof} % proof (to be used with theorem, lemma, proposition, or corollary)
\declaretheorem[style=proofboxn, name=]{replacementproofn} % proof (to be used with theorem, lemma, proposition, or corollary)
\renewenvironment{proof}[1][\proofname]{\def\proofarg{#1}\ifthenelse{\equal{\proofarg}{\proofname}}{\vspace{-12.5pt}\begin{replacementproofn}}{\begin{replacementproof}(\proofarg)}}{\ifthenelse{\equal{\proofarg}{\proofname}}{\end{replacementproofn}}{\end{replacementproof}}\refstepcounter{proof}}

\declaretheorem[numbered=no, style=lightgreenbox, name=Solution]{spacedsolution}
\newenvironment{solution}{\vspace{-12.5pt}\begin{spacedsolution}}{\end{spacedsolution}}

\declaretheorem[style=explanationbox, name=Explanation]{spacedexplanation}
\newenvironment{explanation}{\vspace{-12.5pt}\begin{spacedexplanation}}{\end{spacedexplanation}}

\declaretheorem[numbered=no, style=lightgreenbox, name=Answer]{spacedanswer}
\newenvironment{answer}{\vspace{-12.5pt}\begin{spacedanswer}}{\end{spacedanswer}}

% =================================================================================================

% fixes spacing
\def\thm@space@setup{\thm@preskip=\parskip \thm@postskip=0pt}
\renewcommand{\cftbeforetoctitleskip}{0pt}
\renewcommand{\cftaftertoctitleskip}{0.4\baselineskip}
\renewcommand{\cftbeforelectitleskip}{0.8em}
\renewcommand{\cftafterlectitleskip}{0.5\baselineskip}
\renewcommand{\cftlectitlefont}{\LARGE\bfseries}
\renewcommand{\cfttoctitlefont}{\LARGE\bfseries}

% =================================================================================================

% headers
\pagestyle{fancy}
\fancyhead[LO, RE]{Ray Hang}
\fancyhead[LE, RO]{\@currentlabelname}
\fancyfoot[LE, RO]{\thepage}
\fancyfoot[C]{\leftmark}

% =================================================================================================

\newcommand{\makelecturetoc}{% make 3-column lecture table of contents
    \vspace{-\cftbeforelectitleskip}
    \section*{\cfttoctitlefont Lectures
      \@mkboth{\cfttoctitlefont Lectures}{\cfttoctitlefont Lectures}} \vspace{-\cftafterlectitleskip}
    \begin{multicols}{3}
        \@starttoc{lec}%
    \end{multicols}
    \onecolumn
}

% custom command to create a lecture section
\newcounter{lecturenum}
\newcommand{\lecture}[2]{
    \ifthenelse{\isempty{#2}}
    {\lecturemargin{#1}{1.15em}}
    {\section{#2}\lecturemargin{#1}{-1.75em}}
}

% this should work whenever \lecture does not
\newcommand{\nlecture}[2]{
    \ifthenelse{\isempty{#2}}
    {\lecturenmargin{#1}{1.15em}}
    {\section{#2}\lecturenmargin{#1}{-1.75em}}
}

\newcommand{\lecturemargin}[2]{
    \marginpar[\vspace{#2}\small\itshape#1]{}
    \refstepcounter{lecturenum}
    \addcontentsline{lec}{lectures}{\thelecturenum : #1}
}

\newcommand{\lecturenmargin}[2]{
    \marginnote[\vspace{#2}\small\itshape#1]{}
    \refstepcounter{lecturenum}
    \addcontentsline{lec}{lectures}{\thelecturenum : #1}
}

% custom command to create a tutorial section
\def\@tutorial{}
\newcommand{\tutorial}[2]{
    \ifthenelse{\isempty{#2}}
    {\def\@tutorial{Tutorial #1}}
    {\def\@tutorial{Tutorial #1: #2}}
    \section{\@tutorial}
}

% custom command to create a double hat
\newcommand{\hhat}[1]{% 
    \begingroup%
    \let\macc@kerna\z@%
    \let\macc@kernb\z@%
    \let\macc@nucleus\@empty%
    \hat{\mathchoice%
        {\raisebox{.35ex}{\vphantom{\ensuremath{\displaystyle #1}}}}%
        {\raisebox{.35ex}{\vphantom{\ensuremath{\textstyle #1}}}}%
        {\raisebox{.16ex}{\vphantom{\ensuremath{\scriptstyle #1}}}}%
        {\raisebox{.14ex}{\vphantom{\ensuremath{\scriptscriptstyle #1}}}}%
        \smash{\hat{#1}}}%
    \endgroup%
}

% allows for formatting matrix environments
\renewcommand*\env@matrix[1][*\c@MaxMatrixCols c]{%
    \hskip -\arraycolsep
    \let\@ifnextchar\new@ifnextchar
    \array{#1}}

\makeatother % resets the @ symbol

% aliases
\renewcommand{\arcsin}{\sin^{-1}}
\renewcommand{\arccos}{\cos^{-1}}
\renewcommand{\arctan}{\tan^{-1}}
\newcommand{\arcsec}{\sec^{-1}}
\newcommand{\arccsc}{\csc^{-1}}
\newcommand{\arccot}{\cot^{-1}}
\newcommand{\union}{\cup}
\newcommand{\intersect}{\cap}
\newcommand{\cross}{\times}

% figure support (https://castel.dev/post/lecture-notes-2)
% \usepackage{import}
% \usepackage{pdfpages}
% \usepackage{transparent}
% \newcommand{\incfig}[1]{%
%     \def\svgwidth{\columnwidth}
%     \import{./figures/}{#1.pdf_tex}
% }

% starts chapter counter at specified number
\newcommand{\nchapter}[2]{
    \setcounter{chapter}{#1}
    \addtocounter{chapter}{-1}
    \chapter{#2}
}

% starts section counter at specified number
\newcommand{\nsection}[3]{
    \setcounter{chapter}{#1}
    \setcounter{section}{#2}
    \addtocounter{section}{-1}
    \section{#3}
}

\newcommand{\code}{\lstinline} % inline codeblock
\newcommand{\forcepar}{\hspace{-10px}~\par} % forces a line
\newcommand*{\eval}[3]{\left.#1\right\rvert_{#2}^{#3}} % evaluation bar for integrals and differentials
\newcommand{\overbar}[1]{\mkern 1.5mu\overline{\mkern-1.5mu#1\mkern-1.5mu}\mkern 1.5mu} % boolean logic overline

% default author
\author{Ray Hang}
