% tex-fmt: off

% =================================================================================================

% fonts and language
\usepackage[utf8]{inputenc}
\usepackage[T1]{fontenc}
\usepackage[UKenglish]{babel}
\usepackage[lighttt]{lmodern}

% math packages
\usepackage{mathtools, amsfonts, amssymb, amsthm, latexsym, mathrsfs, stmaryrd} % general math symbols
\usepackage{polynom} % tools for manipulating polynomials
\usepackage{systeme} % creating systems of equations

% tables and figures
\usepackage{array, booktabs, multicol, multirow} % nicer tables
\usepackage{siunitx} % SI units
\usepackage{subcaption} % subfigure captions
\usepackage{nicematrix} % nicer matrices

% graphics and colours
\usepackage[dvipsnames]{xcolor} % colours and coloured text
\usepackage{tikz, pgfplotstable, tikz-cd, circuitikz} % TikZ-based tools
\usepackage[all]{xy} % diagrams

% code and algorithms
\usepackage[ruled, vlined, linesnumbered, titlenumbered]{algorithm2e} % pseudocode for algorithms
\usepackage{listings, lstautogobble} % code listings with autogobble

% text and formatting
\usepackage{bm} % bold math font
\usepackage{parskip} % no indents
\usepackage{cancel} % \cancel command for slashing things
\usepackage[autostyle, english=british]{csquotes} % use literal quotes as quotes
\usepackage{gensymb, textcomp} % generic text symbols
\usepackage[normalem]{ulem} % allows for strike-through text
\usepackage[shortlabels]{enumitem} % more options for enumerate
\usepackage{emptypage} % makes a truly empty page (this is, no headers or footers)
\usepackage{titlesec} % tools for changing titles of sections
\usepackage{tocloft} % allows me to create a new table of contents

% utilities
\usepackage{etoolbox} % tools for creating commands
\usepackage{letltxmacro} % internal macros
\usepackage{xifthen} % extended if then else commands
\usepackage{xstring} % tools for manipulating strings

% margins and layout
\usepackage[hmargin=1.4in,vmargin=1.4in,marginparwidth=0.75in]{geometry} % margins
\usepackage{fancyhdr} % fancy headers
\usepackage{marginnote} % for when \marginpar doesn't work
\usepackage{float} % positioning of floating objects

% hyperlinks
\usepackage[hypertexnames=false]{hyperref} % hyperref but with fixes
\usepackage{bookmark} % hyperref but with fixes
\usepackage{href-ul} % underline hyperlinks

% frames and theorems
\usepackage[framemethod=TikZ]{mdframed} % nicer environments
\usepackage{thmtools} % tools for making theorems

% accessibility for copying and searching
\input{glyphtounicode}

% =================================================================================================

% custom colours
\definecolor{mediumgrey}{rgb}{0.5,0.5,0.5}
\definecolor{lightgrey}{gray}{0.95}
\definecolor{amethyst}{rgb}{0.58,0,0.82}
\definecolor{forestgreen}{HTML}{009900}
\definecolor{royalpurple}{rgb}{0.32, 0.09, 0.98}
\definecolor{sunsetorange}{rgb}{1.0, 0.6, 0.4}
\definecolor{lightblue}{rgb}{0.447, 0.737, 0.831}

% =================================================================================================

% forces PDF version
\pdfminorversion=7
\pdfsuppresswarningpagegroup=1

% =================================================================================================

% margins and text dimensions
\setlength{\textheight}{9in}
\setlength{\textwidth}{6.5in}
\setlength{\topmargin}{0.0in}
\setlength{\leftmargin}{0.0in}
\setlength{\oddsidemargin}{0.0in}
\reversemarginpar

\apptocmd\normalsize{%
	\abovedisplayskip=5pt plus 2pt minus 5pt
	\belowdisplayskip=5pt plus 2pt minus 5pt
}{}{}

% =================================================================================================

% titles, sections, and toc stuff
\newlistof{lectures}{lec}{Lectures} % define the lecture section of the ToC

\setcounter{tocdepth}{3}
\setcounter{secnumdepth}{1}

% =================================================================================================

% makes things appear underneath certain operators by default by appending \limits to them
\apptocmd{\lim}{\limits}{}{}
\apptocmd{\sum}{\limits}{}{}
\apptocmd{\prod}{\limits}{}{}
\apptocmd{\bigwedge}{\limits}{}{}
\apptocmd{\bigvee}{\limits}{}{}
\apptocmd{\bigcap}{\limits}{}{}
\apptocmd{\bigcup}{\limits}{}{}
\apptocmd{\bigoplus}{\limits}{}{}
\apptocmd{\bigotimes}{\limits}{}{}
\apptocmd{\max}{\limits}{}{}
\apptocmd{\min}{\limits}{}{}
\apptocmd{\sup}{\limits}{}{}
\apptocmd{\inf}{\limits}{}{}

% =================================================================================================

% changing the looks of some variables to better looking versions
\let\epsilon\varepsilon
\let\iff\Leftrightarrow
\let\implies\Rightarrow
\let\impliedby\Leftarrow

% =================================================================================================

% set some figure settings
\graphicspath{{./Figures/}}
\pgfplotsset{compat=newest}
\usepgfplotslibrary{fillbetween}
\usetikzlibrary{patterns}
\usetikzlibrary{intersections}

% =================================================================================================

% make the @ symbol a usable letter for the next bit
\makeatletter

% indent subsections in the ToC
\renewcommand*\l@subsection{\@dottedtocline{2}{5.5em}{4.5em}}

% =================================================================================================

% package settings

\lstset{
	autogobble,
	basewidth=0.55em,
	basicstyle={\small\ttfamily},
	belowskip=1mm,
	breakatwhitespace=false,
	breakindent=1em,
	breaklines=true,
	captionpos=b,
	emphstyle={\bfseries\underbar},
	escapeinside={(*}{*)},
	frame=single,
	framexleftmargin=2mm,
	keepspaces=true,
	numbers=left,
	numbersep=4mm,
	numberstyle=\scriptsize\color{mediumgrey},
	postbreak=\raisebox{0ex}[0ex][0ex]{\ensuremath{\hookrightarrow\space}},
	showspaces=false,
	showstringspaces=false,
	showtabs=false,
	tabsize=2,
	upquote=true,
	xleftmargin=4mm,
}

\@ifclasswith{report}{nocolor}
{
	\hypersetup{
		colorlinks,
		linkcolor={black!50!black},
		citecolor={black!50!black},
		urlcolor={black!80!black},
	}
	\lstset{
		backgroundcolor=\color{white},
		commentstyle=\color{mediumgrey},
		keywordstyle=\bfseries,
		stringstyle=\itshape,
	}
}
{
	\hypersetup{
		colorlinks,
		linkcolor={red!50!black},
		citecolor={blue!50!black},
		urlcolor={blue!80!black},
	}
	\lstset{
		backgroundcolor=\color{lightgrey},
		commentstyle=\color{forestgreen},
		keywordstyle=\color{blue},
		stringstyle=\color{amethyst},
	}
}

% =================================================================================================

% custom commands
\@ifclasswith{report}{nocolor}
{
	\newcommand{\red}[1]{\sout{#1}}
	\newcommand{\green}[1]{#1}
}
{
	\newcommand{\red}[1]{\color{red}{#1}}
	\newcommand{\green}[1]{\color{forestgreen}{#1}}
}

\newcommand{\correction}[2]{\ensuremath{\:}{\red{#1}}\ \ensuremath{\to }\ {\green{#2}}\ensuremath{\:}}
\newcommand{\hr}{\par \noindent\rule[0.5ex]{\linewidth}{0.5pt} \par}

% =================================================================================================

% basic margin setup
\mdfsetup{skipabove=12pt, skipbelow=0em, innertopmargin=4pt, innerbottommargin=7pt}

% basic style setup
\theoremstyle{definition}

% custom environments
\@ifclasswith{report}{nocolor}
{
	\declaretheoremstyle[
		headfont=\bfseries\sffamily, bodyfont=\normalfont,
		headformat=\NAME~\NUMBER \NOTE,
		spaceabove=2.5pt,
		mdframed={}
	]{purplebox}
	
	\declaretheoremstyle[
		headfont=\bfseries\sffamily, bodyfont=\normalfont,
		headformat=\NAME \NOTE,
		numbered=no,
		spaceabove=2.5pt,
		mdframed={rightline=false, topline=false, bottomline=false},
	]{purpleline}

	\declaretheoremstyle[
		headfont=\bfseries\sffamily, bodyfont=\normalfont,
		headformat=\NAME \NOTE,
		numbered=no,
		spaceabove=2.5pt,
		mdframed={rightline=false, topline=false, bottomline=false}
	]{orangebox}

	\declaretheoremstyle[
		headfont=\bfseries\sffamily, bodyfont=\normalfont,
		headformat=\NAME \NOTE,
		numbered=no,
		spaceabove=2.5pt,
		mdframed={rightline=false, topline=false, bottomline=false}
	]{orangeline}

	\declaretheoremstyle[
		headfont=\bfseries\sffamily, bodyfont=\normalfont,
		headformat=\NAME \NOTE,
		numbered=no,
		spaceabove=2.5pt,
		mdframed={},
	]{greenbox}

	\declaretheoremstyle[
		headfont=\bfseries\sffamily, bodyfont=\normalfont,
		headformat=\NAME \NOTE,
		numbered=no,
		spaceabove=2.5pt,
		mdframed={rightline=false, topline=false, bottomline=false},
	]{lightgreenbox}

	\declaretheoremstyle[
		headfont=\bfseries\sffamily, bodyfont=\normalfont,
		headformat=\NAME \NOTE,
		numbered=no,
		spaceabove=2.5pt,
		mdframed={},
	]{greenline}

	\declaretheoremstyle[
		headfont=\bfseries\sffamily, bodyfont=\normalfont,
		headformat=\NAME \NOTE,
		numbered=no,
		spaceabove=2.5pt,
		mdframed={}
	]{bluebox}

	\declaretheoremstyle[
		headfont=\bfseries\sffamily, bodyfont=\normalfont,
		headformat=\NAME \NOTE,
		numbered=no,
		spaceabove=2.5pt,
		mdframed={rightline=false, topline=false, bottomline=false}
	]{blueline}

	\declaretheoremstyle[
		headfont=\bfseries\sffamily, bodyfont=\normalfont,
		headformat=\NAME~\NUMBER \NOTE,
		spaceabove=2.5pt,
		mdframed={}
	]{redbox}

	\declaretheoremstyle[
		headfont=\bfseries\sffamily, bodyfont=\normalfont,
		headformat=\NAME~\NUMBER \NOTE,
		spaceabove=2.5pt,
		mdframed={rightline=false, topline=false, bottomline=false},
		qed=\qedsymbol
	]{proofbox}

	\declaretheoremstyle[
		headfont=\bfseries\sffamily, bodyfont=\normalfont,
		headformat=,
		headpunct=,
		postheadspace=0px,
		spaceabove=2.5pt,
		mdframed={rightline=false, topline=false, bottomline=false},
		qed=\qedsymbol
	]{proofboxA}

	\declaretheoremstyle[
		headfont=\bfseries\sffamily, bodyfont=\normalfont,
		headformat=\NAME \NOTE,
		numbered=no,
		spaceabove=2.5pt,
		mdframed={rightline=false, topline=false, bottomline=false},
	]{explanationbox}
}
{
	\declaretheoremstyle[
		headfont=\bfseries\sffamily\color{royalpurple!70!black}, bodyfont=\normalfont,
		headformat=\NAME~\NUMBER \NOTE,
		mdframed={
			linewidth=2pt,
			rightline=false, topline=false, bottomline=false,
			linecolor=royalpurple, backgroundcolor=royalpurple!10,
		}
	]{purplebox}

	\declaretheoremstyle[
		headfont=\bfseries\sffamily\color{royalpurple!70!black}, bodyfont=\normalfont,
		headformat=\NAME \NOTE,
		numbered=no,
		mdframed={
			linewidth=2pt,
			rightline=false, topline=false, bottomline=false,
			linecolor=royalpurple
		}
	]{purpleline}

	\declaretheoremstyle[
		headfont=\bfseries\sffamily\color{sunsetorange!70!black}, bodyfont=\normalfont,
		headformat=\NAME \NOTE,
		numbered=no,
		mdframed={
			linewidth=2pt,
			rightline=false, topline=false, bottomline=false,
			linecolor=sunsetorange, backgroundcolor=sunsetorange!10,
		}
	]{orangebox}

	\declaretheoremstyle[
		headfont=\bfseries\sffamily\color{sunsetorange!70!black}, bodyfont=\normalfont,
		headformat=\NAME \NOTE,
		numbered=no,
		mdframed={
			linewidth=2pt,
			rightline=false, topline=false, bottomline=false,
			linecolor=sunsetorange
		}
	]{orangeline}

	\declaretheoremstyle[
		headfont=\bfseries\sffamily\color{ForestGreen!70!black}, bodyfont=\normalfont,
		headformat=\NAME \NOTE,
		numbered=no,
		mdframed={
			linewidth=2pt,
			rightline=false, topline=false, bottomline=false,
			linecolor=ForestGreen, backgroundcolor=ForestGreen!10,
		}
	]{greenbox}

	\declaretheoremstyle[
		headfont=\bfseries\sffamily\color{ForestGreen!70!black}, bodyfont=\normalfont,
		headformat=\NAME \NOTE,
		numbered=no,
		mdframed={
			linewidth=2pt,
			rightline=false, topline=false, bottomline=false,
			linecolor=ForestGreen, backgroundcolor=ForestGreen!5,
		}
	]{lightgreenbox}

	\declaretheoremstyle[
		headfont=\bfseries\sffamily\color{ForestGreen!70!black}, bodyfont=\normalfont,,
		headformat=\NAME \NOTE,
		numbered=no,
		mdframed={
			linewidth=2pt,
			rightline=false, topline=false, bottomline=false,
			linecolor=ForestGreen
		}
	]{greenline}

	\declaretheoremstyle[
		headfont=\bfseries\sffamily\color{NavyBlue!70!black}, bodyfont=\normalfont,
		headformat=\NAME~\NUMBER \NOTE,
		mdframed={
			linewidth=2pt,
			rightline=false, topline=false, bottomline=false,
			linecolor=NavyBlue, backgroundcolor=NavyBlue!10,
		}
	]{bluebox}

	\declaretheoremstyle[
		headfont=\bfseries\sffamily\color{NavyBlue!70!black}, bodyfont=\normalfont,
		headformat=\NAME \NOTE,
		numbered=no,
		mdframed={
			linewidth=2pt,
			rightline=false, topline=false, bottomline=false,
			linecolor=NavyBlue
		}
	]{blueline}

	\declaretheoremstyle[
		headfont=\bfseries\sffamily\color{RawSienna!70!black}, bodyfont=\normalfont,
		headformat=\NAME~\NUMBER \NOTE,
		mdframed={
			linewidth=2pt,
			rightline=false, topline=false, bottomline=false,
			linecolor=RawSienna, backgroundcolor=RawSienna!10,
		}
	]{redbox}

	\declaretheoremstyle[
		headfont=\bfseries\sffamily\color{RawSienna!70!black}, bodyfont=\normalfont,
		headformat=\NAME~\NUMBER \NOTE,
		mdframed={
			linewidth=2pt,
			rightline=false, topline=false, bottomline=false,
			linecolor=RawSienna, backgroundcolor=RawSienna!5,
		},
		qed=\qedsymbol
	]{proofbox}

	\declaretheoremstyle[
		headfont=\bfseries\sffamily\color{RawSienna!70!black}, bodyfont=\normalfont,
		headformat=,
		headpunct=,
		postheadspace=0px,
		mdframed={
			linewidth=2pt,
			rightline=false, topline=false, bottomline=false,
			linecolor=RawSienna, backgroundcolor=RawSienna!5,
		},
		qed=\qedsymbol
	]{proofboxA}

	\declaretheoremstyle[
		headfont=\bfseries\sffamily\color{NavyBlue!70!black}, bodyfont=\normalfont,
		headformat=\NAME \NOTE,
		numbered=no,
		mdframed={
			linewidth=2pt,
			rightline=false, topline=false, bottomline=false,
			linecolor=NavyBlue, backgroundcolor=NavyBlue!5,
		},
	]{explanationbox}
}

\declaretheorem[style=blueline, name=Note]{mynote} % note
\declaretheorem[style=blueline, name=Remark]{myremark} % remark
\declaretheorem[style=blueline, name=Idea]{myidea} % idea

\declaretheorem[style=greenbox, name=Problem]{myproblem} % problem
\declaretheorem[style=greenbox, name=Question]{myquestion} % problem
\declaretheorem[style=greenbox, name=Recall]{myrecall} % recall

\declaretheorem[style=orangebox, name=Notice]{mynotice} % notice
\declaretheorem[style=orangebox, name=Observation]{myobservation} % observation
\declaretheorem[style=orangebox, name=Investigation]{myinvestigation} % investigation
\declaretheorem[style=orangebox, name=Goal]{mygoal} % goal
\declaretheorem[style=orangebox, name=Summary]{mysummary} % summary

\declaretheorem[style=orangeline, name=Conjecture]{myconjecture} % conjecture
\declaretheorem[style=orangeline, name=Intuition]{myintuitionA} % intuition
\newenvironment{myintuition}{\vspace{-10pt}\begin{myintuitionA}}{\end{myintuitionA}}

\declaretheorem[style=purpleline, name=Notation]{mynotation} % notation
\declaretheorem[style=purpleline, name=Convention]{myconvention} % convention

\declaretheorem[numberwithin=chapter, style=bluebox, name=Example]{myexample} % example
\declaretheorem[numberwithin=chapter, style=bluebox, name=Exercise]{myexercise} % exercise

\declaretheorem[numberwithin=chapter, style=purplebox, name=Definition]{mydefinition} % definition
\declaretheorem[numberwithin=chapter, style=purplebox, name=Algorithm]{myalgorithm} % algorithm

\declaretheorem[numberwithin=chapter, style=redbox, name=Claim]{myclaim} % claim
\declaretheorem[numberwithin=chapter, style=redbox, name=Corollary]{mycorollary} % corollary
\declaretheorem[numberwithin=chapter, style=redbox, name=Lemma]{mylemma} % lemma
\declaretheorem[numberwithin=chapter, style=redbox, name=Proposition]{myproposition} % proposition
\declaretheorem[numberwithin=chapter, style=redbox, name=Theorem]{mytheorem} % theorem

\declaretheorem[numberwithin=chapter, style=proofbox, name=Proof]{myproofA} % proof (to be used with theorem, lemma, proposition, or corollary)
\declaretheorem[numberwithin=chapter, style=proofboxA, name=]{myproofB} % proof (to be used with theorem, lemma, proposition, or corollary)
\newenvironment{myproof}[1][\proofname]{\def\proofarg{#1}\ifthenelse{\equal{\proofarg}{\proofname}}%
	{\vspace{-12.5pt}\begin{myproofB}}{\begin{myproofA}[\proofarg]}}%
	{\ifthenelse{\equal{\proofarg}{\proofname}}{\end{myproofB}}{\end{myproofA}}}%

\declaretheorem[style=lightgreenbox, name=Solution]{mysolutionA}%
\newenvironment{mysolution}{\vspace{-12.5pt}\begin{mysolutionA}}{\end{mysolutionA}}%

\declaretheorem[style=explanationbox, name=Explanation]{myexplanationA}%
\newenvironment{myexplanation}{\vspace{-12.5pt}\begin{myexplanationA}}{\end{myexplanationA}}%

\declaretheorem[style=lightgreenbox, name=Answer]{myanswerA}%
\newenvironment{myanswer}{\vspace{-12.5pt}\begin{myanswerA}}{\end{myanswerA}}%

% =================================================================================================

% fixes spacing
\def\thm@space@setup{\thm@preskip=\parskip \thm@postskip=0pt}
\renewcommand{\cftbeforetoctitleskip}{0pt}
\renewcommand{\cftaftertoctitleskip}{0.4\baselineskip}
\renewcommand{\cftbeforelectitleskip}{0.8em}
\renewcommand{\cftafterlectitleskip}{0.5\baselineskip}
\renewcommand{\cftlectitlefont}{\LARGE\bfseries}
\renewcommand{\cfttoctitlefont}{\LARGE\bfseries}

% =================================================================================================

% headers
\pagestyle{fancy}
\fancyhead[LO, RE]{Ray Hang}
\fancyhead[LE, RO]{\@currentlabelname}
\fancyfoot[LE, RO]{\thepage}
\fancyfoot[C]{\leftmark}

% =================================================================================================

\newcommand{\makelecturetoc}{% make 3-column lecture table of contents
	\vspace{-\cftbeforelectitleskip}
	\section*{\cfttoctitlefont Lectures
	\@mkboth{\cfttoctitlefont Lectures}{\cfttoctitlefont Lectures}} \vspace{-\cftafterlectitleskip}
	\begin{multicols}{3}
		\@starttoc{lec}%
	\end{multicols}
	\onecolumn
}

% custom command to create a lecture section
\newcounter{lecturenum}
\newcommand{\lecture}[2]{
	\ifthenelse{\isempty{#2}}
	{\lecturemargin{#1}{1.15em}}
	{\section{#2}\lecturemargin{#1}{-1.75em}}
}

% this should work whenever \lecture does not
\newcommand{\nlecture}[2]{
	\ifthenelse{\isempty{#2}}
	{\lecturenmargin{#1}{1.15em}}
	{\section{#2}\lecturenmargin{#1}{-1.75em}}
}

\newcommand{\lecturemargin}[2]{
	\marginpar[\vspace{#2}\small\itshape#1]{}
	\refstepcounter{lecturenum}
	\addcontentsline{lec}{lectures}{\thelecturenum : #1}
	\label{Lecture: #1}
	}
	
\newcommand{\lecturenmargin}[2]{
	\marginnote[\vspace{#2}\small\itshape#1]{}
	\refstepcounter{lecturenum}
	\addcontentsline{lec}{lectures}{\thelecturenum : #1}
	\label{Lecture: #1}
}

% custom command to create a tutorial section
\def\@tutorial{}
\newcommand{\tutorial}[2]{
	\ifthenelse{\isempty{#2}}
	{\def\@tutorial{Tutorial #1}}
	{\def\@tutorial{Tutorial #1: #2}}
	\section{\@tutorial}
}

% allows for formatting matrix environments
\renewcommand*\env@matrix[1][*\c@MaxMatrixCols c]{%
	\hskip -\arraycolsep
	\let\@ifnextchar\new@ifnextchar
\array{#1}}

% https://tex.stackexchange.com/questions/246688/scaling-ddots-in-smallmatrix
\LetLtxMacro\orgvdots\vdots
\LetLtxMacro\orgddots\ddots

\DeclareRobustCommand\vdots{%
  \mathpalette\@vdots{}%
}
\newcommand*{\@vdots}[2]{%
  % #1: math style
  % #2: unused
  \sbox0{$#1\cdotp\cdotp\cdotp\m@th$}%
  \sbox2{$#1.\m@th$}%
  \vbox{%
    \dimen@=\wd0 %
    \advance\dimen@ -3\ht2 %
    \kern.5\dimen@
    % remove side bearings
    \dimen@=\wd2 %
    \advance\dimen@ -\ht2 %
    \dimen2=\wd0 %
    \advance\dimen2 -\dimen@
    \vbox to \dimen2{%
      \offinterlineskip
      \copy2 \vfill\copy2 \vfill\copy2 %
    }%
  }%
}
\DeclareRobustCommand\ddots{%
  \mathinner{%
    \mathpalette\@ddots{}%
    \mkern\thinmuskip
  }%
}
\newcommand*{\@ddots}[2]{%
  % #1: math style
  % #2: unused
  \sbox0{$#1\cdotp\cdotp\cdotp\m@th$}%
  \sbox2{$#1.\m@th$}%
  \vbox{%
    \dimen@=\wd0 %
    \advance\dimen@ -3\ht2 %
    \kern.5\dimen@
    % remove side bearings
    \dimen@=\wd2 %
    \advance\dimen@ -\ht2 %
    \dimen2=\wd0 %
    \advance\dimen2 -\dimen@
    \vbox to \dimen2{%
      \offinterlineskip
      \hbox{$#1\mathpunct{.}\m@th$}%
      \vfill
      \hbox{$#1\mathpunct{\kern\wd2}\mathpunct{.}\m@th$}%
      \vfill
      \hbox{$#1\mathpunct{\kern\wd2}\mathpunct{\kern\wd2}\mathpunct{.}\m@th$}%
    }%
  }%
}

\makeatother % resets the @ symbol

% for augmented small matrices
\newcommand{\TMID}{%
	\hspace{-0.1\arraycolsep}%
	\vrule height 6pt depth 0pt%
	\hspace{-0.1\arraycolsep}%
	\vspace{-2pt}%
}
\newcommand{\MID}{%
	\hspace{-0.1\arraycolsep}%
	\vrule height 7.5pt depth 3.5pt%
	\hspace{-0.1\arraycolsep}%
	\vspace{-4.5pt}%
}
\newcommand{\BMID}{%
	\hspace{-0.1\arraycolsep}%
	\vrule height 6.5pt depth 1.5pt%
	\hspace{-0.1\arraycolsep}%
}

% aliases
\renewcommand{\arcsin}{\sin^{-1}}
\renewcommand{\arccos}{\cos^{-1}}
\renewcommand{\arctan}{\tan^{-1}}
\newcommand{\arcsec}{\sec^{-1}}
\newcommand{\arccsc}{\csc^{-1}}
\newcommand{\arccot}{\cot^{-1}}
\newcommand{\union}{\cup}
\newcommand{\intersect}{\cap}
\newcommand{\cross}{\times}

% figure support (https://castel.dev/post/lecture-notes-2)
% \usepackage{import}
% \usepackage{pdfpages}
% \usepackage{transparent}
% \newcommand{\incfig}[1]{%
%     \def\svgwidth{\columnwidth}
%     \import{./figures/}{#1.pdf_tex}
% }

% starts chapter counter at specified number
\newcommand{\nchapter}[2]{
	\setcounter{chapter}{#1}
	\addtocounter{chapter}{-1}
	\chapter{#2}
}

% starts section counter at specified number
\newcommand{\nsection}[3]{
	\setcounter{chapter}{#1}
	\setcounter{section}{#2}
	\addtocounter{section}{-1}
	\section{#3}
}

\newcommand{\code}{\lstinline} % inline codeblock
\newcommand{\forcepar}{\hspace{-10px}~\par} % forces a line
\newcommand*{\eval}[3]{\left.#1\right\rvert_{#2}^{#3}} % evaluation bar for integrals and differentials
\newcommand{\overbar}[1]{\mkern 1.5mu\overline{\mkern-1.5mu#1\mkern-1.5mu}\mkern 1.5mu} % boolean logic overline

% default author
\author{Ray Hang}

% tex-fmt: on
